\documentclass[xetex,svgnames]{scrartcl}

% packages
\usepackage{xltxtra}
\usepackage{polyglossia}
\usepackage{lsalike}
\usepackage{hyperref}
\usepackage{fontspec}
\usepackage{covington}
\usepackage{scrpage2}
\usepackage{qtree}
\usepackage[left=2cm,right=2cm,top=3cm,bottom=3cm]{geometry}
\usepackage{draftwatermark}
\usepackage{hieroglf}
%\usepackage{simpsons}
\usepackage[weather,misc,alpine]{ifsym}
\usepackage{phaistos}
\usepackage{linearb}
\usepackage{dingbat}
\usepackage{pst-node}
\usepackage{colortbl}
\usepackage{alltt}
\usepackage{listings}

% fonts general
\setmainfont[Mapping=tex-text,Scale=1.0]{FreeSans}
\setsansfont[Mapping=tex-text,Scale=1.0]{FreeSans}
\setmonofont{FreeMono}

% special fonts
\newfontfamily\hana{HAN NOM A}
\newfontfamily\hanb{HAN NOM B}
\newfontfamily\sil{Doulos SIL}
\newfontfamily\grk{Aristarcoj}
\newfontfamily\calligraphy{Chinese Calligraphy}
\newfontfamily\gnm{Gnommish}
\newfontfamily\jin{cjkbronze}
\newfontfamily\jiagu{cjkjiagu}
\newfontfamily\xiaozhuan{shuowenxiaozhuan}
\newfontfamily\pur{Purisa}
\newfontfamily\cjcalligraphy{sinocalligraphy}
%\newfontfamily\ahd{eufm10}

% specific commands
\newcommand{\mysub}[1]{\raisebox{-0.5ex}{\scriptsize{#1}}}
\SetWatermarkText{PYTHON}
\newcommand{\bild}[2]{%
    \scalebox{#1}{%
        \includegraphics{/home/mattis/projects/graphics/img/#2.jpg}
        }
    }
\newcommand{\Table}[1]{%
    \begin{flushleft}
        \begin{tabular}{|p{16.5cm}|}
            \hline \cellcolor{lightgray} \bf \pur #1
            \\\hline
        \end{tabular}
    \end{flushleft}
}

\lstset{language=Python}

\newcommand{\Code}[1]{%
    \begin{flushleft}
        \begin{tabular}{||p{16.5cm}||}
            \hline\hline \cellcolor{white}
            \\
            #1

             \cellcolor{white}
            \\\hline\hline
        \end{tabular}
    \end{flushleft}
}
\newcommand{\White}[1]{\cellcolor{white} \textcolor{black}{ #1}}

% language settings
\setmainlanguage[spelling=new]{german}
\setotherlanguage{english}

% pagestyle settings
\pagestyle{scrheadings}
\ihead{Johann-Mattis List}
\chead{Python und JavaScript}
\ohead{Lösungen 1}
\ifoot{}
\cfoot{\pagemark}
\ofoot{}

%\include{latex/figures}
\lstset{language=perl}
\begin{document}
%\maketitle
\begin{center}
  \bf \huge Praktisches zu Python
\end{center}
``super sie haben alle python bibliotheken richtig installiert!

\begin{center}
  \bf \huge Praktisches zu JavaScript
\end{center}
Sie ein Lösungsbeispiel unter
\url{https://github.com/LinguList/pyjs-seminar/blob/master/website/code/hallo_welt}.

\begin{center}
    {\bf \huge  Jamie Olivers Bratäpfel}
\end{center}
\section*{\textcolor{Crimson}{Aufgabe 1} }
Die folgenden Sätze (genannt werden Schlüsselwörter) sind auf jeden Fall nicht eindeutig verständlich für Maschinen (über
Feinheiten lässt sich streiten): 
\begin{itemize}
        \item {\bf Teil 2:} 
            \begin{itemize}
                \item[1] ``etwas weich''
                \item[3] ``vorsichtig''
                \item[5] ``kleine Stücke'', ``fein''
                \item[7] ``Großteil''
                \item[8] ``gut vermischen und verkneten''
                \item[10] Missverständnis resultiert aus fehlender Präzision in
                    Punkt 7.
            \end{itemize}
        \item {\bf Teil 3:}
            \begin{itemize}
                \item[1] Wie lange genau?
                \item[2] ``etwa 5 Minuten"
                \item[4] Kann eine Maschine nicht beurteilen.
            \end{itemize}
    \end{itemize}
\section*{\textcolor{Crimson}{Aufgabe 2} }
Ich liste hier wieder nur stichwortartig auf, was ich denke, das man wissen
muss, ohne Anspruch auf totale Vollständigkeit.
\begin{itemize}
    \item ``Vorheizen" >> ``heiß"
    \item ``Ofen"
    \item ``braten"
    \item ``goldbraun" >> Farbskalen
    \item ``weich"
    \item ``Apfel"
    \item ``Servieren"
        \item ``abkühlen" >> ``kalt"
        \item ``klein'' in Bezug auf ``Schalen"
        \item ``Saft"
        \item \ldots
    \end{itemize}
\section*{\textcolor{Crimson}{Aufgabe 3} }
Müsste wohl gehen! Die Zutaten, bei denen Varianten möglich sind, können als
Schlüsselwörter übergeben werden. Die Mengenangaben können als Verhältnisse von der
Personenzahl, die ich als Argument angeben wird, abgeleitet werden. Was genau
intern geschieht, muss dann noch mal explizit erörtert werden.
Die Funktion kann dann wie folgt aufgerufen werden: BratAepfel(Personen=4),
womit Bratäpfel für vier Personen gebacken werden.
\section*{\textcolor{Crimson}{Aufgabe 4} }
\begin{lstlisting}[frame=trLB,numbers=left,numberstyle=\tiny]
# Wir ermitteln erst mal die Startzeit
$StartZeit = time('now') 
# Jetzt lassen wir die Butter weich werden
ButterWeichWerdenLassen( )
# Jetzt heizen wir den Backofen vor
BackofenVorheizen( )
# Jetzt beginnen wir mit den Apfeloperationen
for Apfel in ApfelMenge:
do
    KernGehaeuseEntfernen(Apfel)
    MitEinemMesserInDerMitteRundumEinschneiden(Apfel)
    InEineOfenfesteFormSetzen(Apfel)
# Jetzt schneiden wir die Lorbeerblätter
if Zustand(Lorbeerblaetter) == 'getrocknet'
do 
    InKleineStueckeReissen(Lorbeerblaetter)
else if Zustand(Lorbeerblaetter) == 'frisch'
do
    FeinHacken(Lorbeerblaetter)
# usw. usf. 
# Die Lösung ist zugegebenermaßen etwas faul und grob. Es geht besser...

\end{lstlisting}


\end{document}

