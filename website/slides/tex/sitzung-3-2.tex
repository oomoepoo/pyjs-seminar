\hyperdef{}{tilda}{}

\subsection{Automatische Sequenzalinierung}

\subsubsection{\texorpdfstring{{Das Problem}}{Das Problem}}

\par\noindent\textbf{Wir wollen zwei Sequenzen automatisch alinieren, aber wie?}

\begin{itemize}
\itemsep1pt\parskip0pt\parsep0pt
\item
  {Eine einfache automatische Lösung wäre es, einfach alle verschiedenen
  Kombinationen durchzutesten.}
\item
  {In einem weiteren Schritt könnten wir dann alle Alinierungen testen
  und ihren ``Score'' berechnen, also sie bewerten, indem wir, zum
  Beispiel, zählen, wie viele Mismatches und leere Matches sie
  aufweisen.}
\item
  {Dann könnten wir den Score für jedes Paar aufsummieren und würden
  einfach die Alinierung mit dem besten Score nehmen.}
\end{itemize}




\par\noindent\textbf{Leider dauert das zu lange!}

Die Zahl aller Möglichen Alinierungen zwischen zwei Strings hängt von
deren Länge ab: Es gilt:

\$N = \textbackslash{}sum\_\{k=0\}\^{}\{\textbackslash{}min(m,n)\}
2\^{}\{k\} \textbackslash{}cdot \textbackslash{}binom\{m\}\{k\}
\textbackslash{}cdot \textbackslash{}binom\{n\}\{k\}\$,

für zwei Strings der Länge \$m\$ und \$n\$.




\par\noindent\textbf{Und was soll das heißen?}

Das wir für die Alinierung von ``Herz'' und ``heart'' 681 verschiedene
Alinierungen testen müssten, und für die Alinierung von ``Werdegänge''
und ``Kinderpass'' 8 097 453 verschiedene Möglichkeiten!



\subsubsection{\texorpdfstring{{Die
Lösungsstrategie}}{Die Lösungsstrategie}}

\par\noindent\textbf{Sei faul und mach Dir keine unnötige Arbeit!}

Die Lösungsstrategie beruht auf der Idee, dass man bestimmte
Teillösungen, die man schon berechnet hat, ja eigentlich nicht noch ein
zweites Mal berechnen muss, und ferner, dass man Lösungswege, die so
abwegig sind, dass klar ist, dass sie nicht zum Erfolg führen können,
einfach nicht weiterverfolgt.




\par\noindent\textbf{Dynamische Programmierung}

Der Algorithmus, mit dem man die optimale Alinierung von zwei Strings
relativ schnell und effizient berechnen kann, gehört zur Familie der
dynamischen Programmieralgorithmen (\emph{dynamic programming}). Die
Idee dieser Algorithmen ist es, ein Problem in umgekehrter Richtung
anzugehen: Anstatt alle möglichen Alinierungen zwischen zwei Sequenzen
zu testen, baut man eine Alinierung auf, indem man Gebrauch macht von
``previous solutions for optimal alignments of smaller subsequenze''
(\href{http://bibliography.lingpy.org?key=Durbin2002}{Durbin et al.
1998{[}2002{]}}).



\subsubsection{Der Algorithmus}

\par\noindent\textbf{Grundlegende Bestandteile}

\begin{enumerate}
\itemsep1pt\parskip0pt\parsep0pt
\item
  \textbf{Scoring Schema:} bestimmt, wie wir uniforme, divergente, und
  leere Matches bewerten.
\item
  \textbf{Matrizzenkonstruktion:} erstellt eine Matrix, in der wir alle
  möglichen Kombinationen der Sequenzen miteinander einander
  gegenüberstellen.
\item
  \textbf{Traceback:} Merkt sich alle Zwischenentscheidungen, die
  getroffen wurden, so dass wir nachher ermitteln können, welchen
  ``Pfad'' wir gelaufen sind, umz um besten Ergebnis zu gelangen.
\end{enumerate}





\par\noindent\textbf{Scoring Schema}

\begin{longtable}[c]{@{}lll@{}}
\toprule
Matching Type & Score & Example\tabularnewline
\midrule
\endhead
uniform match & 0 & A / A\tabularnewline
divergent match & 1 & A / B\tabularnewline
emtpy match & 1 & A / -, - / A\tabularnewline
\bottomrule
\end{longtable}

\par\noindent\textbf{Matrix}

\begin{tabular}[c]{|l|l|l|l|}
\hline
  - / - & - / B & - / Ä & - / R\\\hline
A / - & A / B & A / Ä & A / R\\\hline
B / - & B / B & B / Ä & B / R\\\hline
E / - & E / B & E / Ä & E / R\\\hline
R / - & R / B & R / Ä & R / R\\\hline
\end{tabular}









\par\noindent\textbf{Traceback}

\begin{longtable}[c]{@{}llll@{}}
\toprule
- / - & - / B & - / Ä & - / R\tabularnewline
A / - & A / B & A / Ä & A / R\tabularnewline
B / - & B / B & B / Ä & B / R\tabularnewline
E / - & E / B & E / Ä & E / R\tabularnewline
R / - & R / B & R / Ä & R / R\tabularnewline
\bottomrule
\end{longtable}





\par\noindent\textbf{Traceback}






\subsection{\texorpdfstring{{Sequenzalinierung in
JavaScript}}{Sequenzalinierung in JavaScript}}

\subsubsection{\texorpdfstring{{Vorüberlegungen}}{Vorüberlegungen}}

\par\noindent\textbf{Um den Algorithmus zu implementieren, brauchen wir:}

\begin{itemize}
\itemsep1pt\parskip0pt\parsep0pt
\item
  einige numerische Variablen
\item
  die Repräsentation der Sequenzen
\item
  Wege, Alinierungen zu repräsentieren
\item
  Wege, die Matrix zu repräsentieren
\item
  eine Möglichkeit, das Scoring-Schema umzusetzen
\end{itemize}



\par\noindent\textbf{JavaScript bietet uns:}

\begin{itemize}
\itemsep1pt\parskip0pt\parsep0pt
\item
  Variablen, kein Problem
\item
  Listen (Arrays) für die Alinierungen
\item
  und auch für die Sequenzen
\item
  mehrdimensionale Arrays für die Matrix und den Traceback
\item
  if/else-Verzweigungen und String-Vergleiche für das Scoring-Schema
\end{itemize}


\subsubsection{\texorpdfstring{{Implementierung der
Matrixberechnung}}{Implementierung der Matrixberechnung}}

\begin{verbatim}
function wf_align(seqA, seqB) {
  /* Vorbereitung der Daten hier */
  /* Matrix-Initialisierung hier */
  /* Haupt-Loop hier, darin auch die Scoring Function */
  /* Traceback hier */
  /* Nachbearbeitung hier */
}
\end{verbatim}



\begin{verbatim}
function wf_align(seqA, seqB) {
  /* return nothing if either of the lists is empty */
  if(seqA.length == 0 || seqB.length == 0) {
    return;
  }
  /* get the lengths of the sequences */
  var alen = seqA.length;
  var blen = seqB.length;
  /* declare variables in local environment */
  var i, j; // numbers
  var gapA, gapB, dist; // floats
  var charA, charB; // characters
  /* Matrix-Initialisierung hier */
  /* Haupt-Loop hier, darin auch die Scoring Function */
  /* Traceback hier */
  /* Nachbearbeitung hier */
}
\end{verbatim}


\begin{verbatim}
function wf_align(seqA, seqB) {
  /* Vorbereitung hier */
  /* create the matrix */
  var matrix = [];
  for(var i=0;i<alen+1;i++) {
    var inline = [];
    for(var j=0;j<blen+1;j++) {
      inline.push(0);
    }
    matrix.push(inline);
  }
  /* initialize matrix */
  for(i=1;i<blen+1;i++) {
    matrix[0][i] = i;
  }
  for(i=1;i<alen+1;i++) {
    matrix[i][0] = i;
  }
  /* create the traceback */
  var traceback = [];
  for(var i=0;i<alen+1;i++) {
    var inline = [];
    for(var j=0;j<blen+1;j++) {
      inline.push(0);
    }
    traceback.push(inline);
  }
  /* initialize traceback */
  for(i=1;i<blen+1;i++) {
    traceback[0][i] = 2;
  }
  for(i=1;i<alen+1;i++) {
    traceback[i][0] = 1;
  }
  /* Haupt-Loop hier, darin auch die Scoring Function */
  /* Traceback hier */
  /* Nachbearbeitung hier */
}
\end{verbatim}



\begin{verbatim}
function wf_align(seqA, seqB) {
  /* Vorbereitung hier */  
  /* Matrix-Initialisierung hier */ 
  /* start the iteration to fill the matrix */
  for(i=1;i<alen+1;i++) {
    for(j=1;j<blen+1;j++) {      
      /* get the character in both sequences at their respective position */
      charA = seqA[i-1];
      charB = seqB[j-1];      
      /* check the similarity between the characters to get the local distance */
      if(charA == charB) {
        dist = matrix[i-1][j-1];
      }
      else {
        dist = matrix[i-1][j-1]+1;
      }      
      /* we have the distance for substitution, now we need the gaps */
      gapA = matrix[i-1][j]+1;
      gapB = matrix[i][j-1]+1;    
      /* find the minimal value */
      if(dist < gapA && dist < gapB) {
        matrix[i][j] = dist;
      }
      else if(gapA < gapB) {
        matrix[i][j] = gapA ;
        traceback[i][j] = 1;
      }
      else {
        matrix[i][j] = gapB;
        traceback[i][j] = 2;
      }
    }
  }
  /* Traceback hier */
  /* Nachbearbeitung hier */
}
\end{verbatim}


\subsubsection{\texorpdfstring{{Implementierung des
Traceback}}{Implementierung des Traceback}}

\begin{verbatim}
function wf_align(seqA, seqB) {
  /* Vorbereitung hier */
  /* Matrix-Initialisierung hier */
  /* get indices for the last cells of the matrix */
  i = matrix.length-1;
  j = matrix[0].length-1;
  /* get the edit-dist */
  var ED = matrix[i][j];
  /* initialize the alignment arrays */
  var almA = [];
  var almB = [];
  /* start the traceback */
  while(i > 0 || j > 0) {
    if(traceback[i][j] == 0) {
      almA.push(seqA[i-1]);
      almB.push(seqB[j-1]);
      i--;
      j--;
    }
    else if(traceback[i][j] == 1) {
      almA.push(seqA[i-1]);
      almB.push("-");
      i--;
    }
    else {
      almA.push("-");
      almB.push(seqB[j-1]);
      j--;
    }   
  }
  /* Nachbearbeitung hier */
}
\end{verbatim}



\begin{verbatim}
function wf_align(seqA, seqB) {
  /* Vorbereitung hier */
  /* Matrix-Initialisierung hier */
  /* Haupt-Loop hier, darin auch die Scoring Function */
  /* reverse alignments */
  almA = almA.reverse();
  almB = almB.reverse();
}
\end{verbatim}

\subsection{\texorpdfstring{{Interaktive
Sequenzalinierung}}{Interaktive Sequenzalinierung}}

\subsubsection{\texorpdfstring{{Grundlagen}}{Grundlagen}}

\par\noindent\textbf{Input-Felder in HTML}

Um eine interaktive Alinierungsapp zu erstellen, machen wir Gebrauch von
den Input-Felder in HTML. Diese sind sehr einfach zu erstellen:

\begin{verbatim}
\end{verbatim}

Beachten Sie dabei, dass wir unbedingt eine ID als Tag vergeben müssen,
wie auch eine Funktion, die wir diesmal nicht per
\texttt{onclick="function()"}, sondern mittels
\texttt{onkeyup="function()"} ausführbar machen. Das heißt nichts
anderes, als dass jedes mal, wenn wir eine Taste drücken, während wir
mit dem Cursor im Text-Feld sind, potentiell eine Funktion auslösen.



\par\noindent\textbf{Es kann losgehen!}

Abgesehen davon brauchen wir eigentlich nur noch die klassische
Einbindung von JavaScript in eine HTML-Seite vorzunehmen. Wir nutzen
dafür das Script
\href{https://github.com/LinguList/pyjs-seminar/blob/master/website/code/nw.js}{nw.js},
welches zuvor besprochen wurde. Auch eine kleine CSS-Datei wird
erstellt, damit es nachher ein wenig besser aussieht. Darauf wird in
diesem Zusammenhang aus Zeitgründen nicht genauer eingegangen), die
findet sich aber online und auf GitHub und kann dort gezielt eingesehen
werden.


\subsubsection{\texorpdfstring{{Implementierung}}{Implementierung}}

\par\noindent\textbf{Der Header}

\begin{verbatim}
<html>
<head>
  <title>Alignment Demo</title>
  <meta http-equiv="content-type" content="text/html; charset=utf-8">
  <script src="js/nw.js" type="text/javascript"></script>
  <link rel="stylesheet" type="text/css" href="css/nw.css" />
</head>
\end{verbatim}



\par\noindent\textbf{Der Body}

\begin{verbatim}
<body>
  <input type="text" style="width: 300px" id="seqA" onkeyup="alignit()" />
  <input type="text" style="width: 300px" id="seqB" onkeyup="alignit()" />
  <div id="display"></div>
<!-- Java-Script-Code kommt hier -->
</body>
\end{verbatim}



\par\noindent\textbf{Der JavaScript-Code}

\begin{verbatim}
function alignit() {
  /* get the sequences */
  var seqA = document.getElementById('seqA');
  var seqB = document.getElementById('seqB');
  /* get the values */
  seqA = seqA.value.split('');
  seqB = seqB.value.split('');
  /* return if one of the vals is none */
  if (seqA == '' || seqB == '') {
    document.getElementById('display').innerHTML = '';
    return;
  }
  /* get the alignment */
  alms = wf_align(seqA, seqB);
  var almA = alms[0];
  var almB = alms[1];
  var dist = alms[2];
  /* create the text */
  var txt = '';
  txt += ''+almA.join('')+'';
  txt += ''+almB.join('')+'';
  txt += 'Edit-Distance: '+dist+'';
  txt += '';
  /* update */
  document.getElementById('display').innerHTML = txt;
}
\end{verbatim}


\subsubsection{\texorpdfstring{{Demo}}{Demo}}

