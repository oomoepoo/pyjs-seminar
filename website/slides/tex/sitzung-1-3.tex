\hyperdef{}{tilda}{}

\subsection{\texorpdfstring{{Allgemeines zu
JavaScript}}{Allgemeines zu JavaScript}}

\subsubsection{\texorpdfstring{{Herkunft}}{Herkunft}}

\href{http://de.wikipedia.org/wiki/JavaScript}{JavaScript}

\begin{itemize}
\itemsep1pt\parskip0pt\parsep0pt
\item
  {ist eine Skriptsprache, die speziell für dynamisches HTML in
  Webbrowsern entwickelt wurde}
\item
  {verändert Inhalte von Webseiten, reagiert auf Benutzereingaben, und
  erweitert die Möglichkeiten von HTML und CSS}
\item
  {ist ein abtrünniges Kind von C, was die hässliche Syntax angeht und
  hat mit Java selbst sehr wenig zu tun (obwohl die Syntax von Java auch
  sehr hässlich ist)}
\item
  {wurde ursprünglich von Netscape entwickelt und im Jahre 1996 in der
  Version 1.1 veröffentlicht}
\end{itemize}


\subsubsection{\texorpdfstring{{Charakteristik}}{Charakteristik}}

\href{http://de.wikipedia.org/wiki/JavaScript}{JavaScript}

\begin{itemize}
\itemsep1pt\parskip0pt\parsep0pt
\item
  {hat eine sehr hässliche Syntax}
\item
  {ist prinzipiell nicht sehr schwer zu erlernen, sieht aber hässlich
  aus}
\item
  {erlaubt es den Benutzern, sehr, sehr hässlichen Kode zu schreibne}
\item
  {ist am Anfang sehr gewöhungsbedürftig, bis man begriffen hat, wie die
  Interaktion zwischen HTML, CSS, und JavaScript abläuft}
\item
  {ist generell ``innen pfui, außen hui''}
\item
  {macht des dem Benutzer sehr schwer, modularen Kode zu schreiben}
\end{itemize}


\subsubsection{\texorpdfstring{{Installation}}{Installation}}

{ JavaScript ist fester Bestandteil gängiger Webbrowser, wie
\href{http://de.wikipedia.org/wiki/Mozilla_Firefox}{Firefox},
\href{http://de.wikipedia.org/wiki/Google_Chrome}{Chrome}, oder
\href{http://de.wikipedia.org/wiki/Apple_Safari}{Safari}. Im
\href{http://de.wikipedia.org/wiki/Internet_Explorer}{Internet-Explorer}
ist JavaScript angeblich auch vorinstalliert, jedoch laufe viele Apps
sehr viel schlechter als gewünscht, weil Microsoft sich mal wieder
qualitativ von den anderen Anbietern abgrenzen muss. }

\subsection{\texorpdfstring{{Bibliotheken und
Entwicklertools}}{Bibliotheken und Entwicklertools}}

\subsubsection{\texorpdfstring{{Webbrowser}}{Webbrowser}}

{ Ich empfehle, beim Entwickeln von JavaScript-Applikationen
grundsätzlich auf einen der gängigen Unix-Browser
(\href{http://de.wikipedia.org/wiki/Mozilla_Firefox}{Firefox},
\href{http://de.wikipedia.org/wiki/Google_Chrome}{Chrome}, oder
\href{http://de.wikipedia.org/wiki/Apple_Safari}{Safari})
zurückzugreifen. Firefox und Chrome, welche ich beide regelmäßig
verwende, unterscheiden sich in den Funktionen, die sie bieten. Firefox
erlaubt bestimmte Dateizugriffe, ohne die Applikation über einen Server
laufen zu lassen, was die Entwicklung von Applikationen erleichtert.
Chrome hat Vorteile im Layout und im Debugging. Grundsätzlich sollten
alle Applikationen mindestens auf Firefox und Chrome getestet werden, da
es hier mitunter zu bestimmten Unterschieden kommen kann. Auch
unerwartete Fehler können in einem Webbrowser auftauchen, im anderen
aber nicht. }


\subsubsection{\texorpdfstring{{Bibliotheken}}{Bibliotheken}}

\par\noindent\textbf{Empfohlene Frameworks:}

\begin{itemize}
\itemsep1pt\parskip0pt\parsep0pt
\item
  {\href{http://jquery.com}{jQuery}: eine der am weitesten verbreiteten
  JavaScript-Bibliotheken, die viele Erweiterungen und Erleichterungen
  bietet und als Grundlage zahlreicher Plugins dient. Aufgrund der Größe
  von jQuery sollte man sich jedoch für jedes Projekt überlegen, ob man
  die Bibliothek auch wirklich braucht, denn auch JavaScript allein
  bietet viele Möglichkeiten, kurzen und knackigen Kode zu schreiben.}
\item
  {\href{http://d3js.org}{d3}: Eine wunderbare Bibliothek zur
  Datenvisualisierung, ein bisschen gewöhnungsbedürftig in der
  Anwendung, aber unheimlich schön in den Ergebnissen. Entwickelt wurde
  die Bibliothek vorrangig von Mike Bostock, der für die New York Times
  arbeitet, die seine Pionierarbeit der interaktiven Visualisierung
  sponsort.}
\end{itemize}


\subsubsection{\texorpdfstring{{Entwicklertools}}{Entwicklertools}}

Man kann JavaScript auch in der Konsole ausführen (die selbst in
JavaScript geschrieben wurde):

{ Zum Testen bietet sich \href{http://nodejs.org}{node.js} an: Dieses
Paket erlaubt es, JavaScript Kode in Skripten auszuführen und bietet
auch eine interaktive Konsole. Der Vorteil von node.js ist, dass man
Skripte testen kann, ohne den ansonsten erforderlichen HTML/CSS-Überbau
zugrunde legen zu müssen. }

\subsection{\texorpdfstring{{Ein erstes
Programmierbeispiel}}{Ein erstes Programmierbeispiel}}

\subsubsection{\texorpdfstring{{Die Kölner Phonetik in
JavaScript}}{Die Kölner Phonetik in JavaScript}}

{ Es existiert natürlich bereits eine Implementierung der Kölner
Phonetik für JavaScript, implementiert von J. Tillmann:
\url{https://github.com/jtillmann/colophoneticjs}. Diese Version wurde
für die Zwecke unseres Seminars leicht umgeschrieben, so dass wir sie in
ähnlicher Weise, wie die Python-Version der Kölner Phonetik verwenden
können. Der resultierende Source Code, ein Skript mit Namen
\href{https://github.com/LinguList/pyjs-seminar/blob/master/website/code/kph.js}{kph.js}
befindet sich auf der Projektwebseite und kann dort heruntergeladen
werden. Er wurde ebenfalls bereits in die Terminal-Applikation
integriert. }



\par\noindent\textbf{Kölner Phonetik mit Hilfe des
\href{../demos/console.html}{Web-Terminals}:}

\begin{verbatim}
js> kph.encode('Mayer');
67
\end{verbatim}

{ }



\par\noindent\textbf{Kölner Phonetik mit node.js:}

{Einbinden von kph.js:}

\begin{verbatim}
> kph = require('./js/kph.js') /* Im Order demos/ */
{ encode: [Function] }
\end{verbatim}

{Anwenden:}

\begin{verbatim}
> kph.encode('Müller-Lüdenscheidt');
'65752682'
> kph.encode('Muller-Ludenscheidt Sebastian Bach')
65752682 81826 14
\end{verbatim}



Die Verwendung in der Konsole ist aber noch nicht alles! Viel
interessanter wird es ja, wenn man den Kode gleich auf einer Webseite
einsetzen kann, zum Beispiel als interaktive Applikation.

Dafür brauchen wir zunächst eine HTML-Eingabe und Ausgabe, die wir in
einer HTML-Seite (mit Standard-Head-Body Struktur) wie folgt
unterbringen können:

\begin{verbatim}

OK</>
\end{verbatim}

{Die Ausgabe sieht dann auf der Webseite wie folgt aus:}

OK

\hyperdef{}{opt}{}



Um eine bestimmte JavaScript-Bibliothek ausführen zu können, müssen wir
sie im Header der HTML-Datei (oder an einer beliebigen anderen Stelle)
einbinden, was wie folgt aussieht:

\begin{verbatim}
<html>
  <head>
    
    
  </head>
  <body>
    
    GIB MIR DIE KPH!
    
  </body>
</html>
\end{verbatim}



Dann brauchen wir noch eine JavaScript Funktion, die mit HTML
kommuniziert:

\begin{verbatim}
// Funktion greift auf die HTML-Datei zu und Berechnet
// die Werte für die Kölner Phonetik
function showKPH() {
  /* get the input button element */
  var ipt = document.getElementById('ipt');
  /* get the input value */
  var ipt_val = ipt.value;
  /* convert to Kölner Phonetik */
  var converted_values = kph.encode(ipt_val);
  /* write to html page */
  var opt = document.getElementById('opt');
  opt.innerHTML = converted_values;
}
\end{verbatim}

{Beachten Sie, dass es zwei Arten in JS gibt, um Kommentare einzufügen,
den Doppelslash (//), der den Rest einer Zeile auskommentiert, und die
Kombination von Slash mit Asterisk (/* und */), mit denen man auch über
Zeilen hinaus auskommentieren kann. }



Diese Datei können wir entweder als separates Skript abspeichern und in
HTML einbinden,

\begin{verbatim}
<html>
  ...
    
    
  </body>
\end{verbatim}

oder wir können sie direkt zwischen die Skript-Tags schreiben

\begin{verbatim}
<html>
  ...
    
    
function showKPH() {
  var ipt = document.getElementById('ipt');
  var ipt_val = ipt.value;
  var converted_values = kph.encode(ipt_val);
  var opt = document.getElementById('opt');
  opt.innerHTML = converted_values;
}
    
  </body>
\end{verbatim}



Und so sieht das Ganze dann in Aktion aus:

