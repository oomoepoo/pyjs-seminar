\subsection{\texorpdfstring{{Darstellung geographischer
Daten}}{Darstellung geographischer Daten}}

\subsubsection{\texorpdfstring{{GeoJson und
TopoJson}}{GeoJson und TopoJson}}

\paragraph{Grundlegendes}

\href{http://geojson.org}{GeoJson} und
\href{https://github.com/mbostock/topojson}{TopoJson} sind Formate zur
einfachen Beschreibung geographischer Daten. Sie bauen beide auf dem
JSON-Format auf, welches wir im Laufe des Woche ja schon zuweilen
verwendet haben. Beide Formate stellen eine ideale Grundlage für
geographische Darstellungen mit JavaScript dar und sind auch ineinander
überführbar. Wir werden TopoJson für unser Beispiel verwenden.



\paragraph{GitHub-Integration}

\begin{itemize}
\item
  Das tolle an Geo- und TopoJson ist, dass die Formate direkt in GitHub
  als Karten dargestellt werden.
\item
  Wenn wir uns Beispielsweise die Datei
  \href{https://github.com/LinguList/pyjs-seminar/blob/master/website/demos/china/maps/zh-mainland-provinces.topo.json}{zh-mainland-provinces-topo.json}
  direkt auf GitHub anschauen, dann sehen wir direkt eine Karte, und
  auch, was auf der Karte zu sehen ist!
\end{itemize}


\paragraph{Getting Started}

Grundlegendes zu den Formaten, aber auch zum Gestalten einer Karte kann
man im Internet in vielen Beispielen finden. Ich selbst habe meine
ersten Erkenntnisse über D3 und die Verwendung von geographischen Daten
in JS durch das Beispiel \href{http://bost.ocks.org/mike/map/}{Let's
Make a Map} von Mike Bostock gesammelt.

Dort beschreibt Bostock auch, wo man gute geographischen Daten finden
kann. Diese liegen meist in Form von ``shapefiles'' vor, die nicht in JS
verwendet werden können. Aber es gibt mit
\href{http://mapshaper.org/}{mapshaper} ein sehr gutes Tool, um die
Daten direkt in die JSON-Formate umzuwandeln.


\subsubsection{\texorpdfstring{{D3}}{D3}}

\href{http://d3js.org}{D3} ist eine wunderbare Bibliothek, um Daten zu
Visualisieren und interaktive Applikationen zu schreiben. Leider ist sie
auch sehr kompliziert und für Anfänger nur schwer zugänglich, da sich
hinter dem Kode viele sehr sinnvolle aber auch eigenwillige Konzepte
verbergen.

Wir haben in diesem Seminar leider keine Zeit, voll auf D3 und die
Möglichkeiten einzugehen. Wir werden uns daher auf ein Beispiel
beschränken, in dem es weniger um den Kode als um die Idee der
Visualisierung geht.

Denjenigen, die mehr über D3 erfahren möchte, empfehle ich, mit der
\href{http://d3js.org}{offiziellen Homepage} anzufangen, und sich dann
langsam ``hochzuarbeiten''.

\subsection{\texorpdfstring{{Eine
Beispielapplikation}}{Eine Beispielapplikation}}

\subsubsection{\texorpdfstring{{Idee}}{Idee}}

\paragraph{Die Daten}

Wir haben uns schon mit der Kollektion chinesischer Daten befasst und
sie sogar bereits mit Hilfe von LingPy aliniert. Nun wollen wir einen
Schritt weiter gehen, und die Diversität der Sprachen nicht nur über die
einzelnen Alinierungen, sondern auch im Raum darzustellen.


\subsubsection{\texorpdfstring{{Idee}}{Idee}}

\paragraph{Die Idee}

Die Idee ist einfach: Wir plotten alle Dialektpunkte auf eine Karte, und
erlauben dann den Benutzern, Konzeptweise Daten aufzurufen und bei Klick
auf einen Dialektpunkt sowohl die Verteilung des Kognatensets als auch
die Wörter in alinierter Form zu betrachten.



\paragraph{Die Tools}

\begin{itemize}
\itemsep1pt\parskip0pt\parsep0pt
\item
  Python: zur Alinierung der Daten und zur Aufbereitung der Daten in
  JavaScript-kompatible Formate (JSON, JS Objekte).
\item
  JavaScript: zum Plotten der Daten mit Hilfe von D3.
\end{itemize}


\subsubsection{\texorpdfstring{{Vorbereitung}}{Vorbereitung}}

\paragraph{Programmarbeit}

\begin{itemize}
\itemsep1pt\parskip0pt\parsep0pt
\item
  Erstellen der Daten und Überführung in LingPy-Wortlisten-Format
\item
  Durchführung der Alinierungsanalyse mit Hilfe von LingPy
\item
  Exportieren der Daten in JS-Formate (hauptsächlich JS-Objects, die ja
  wie ein Hash verwendet werden können und identisch mit JSON-Objekten
  sind).
\end{itemize}



\paragraph{Hintergrundarbeit}

\begin{itemize}
\itemsep1pt\parskip0pt\parsep0pt
\item
  Geographische Files konnten übernommen werden aus einem
  \href{https://github.com/clemsos/d3-china-map}{GitHub-Repository}.
\item
  Der Kode musste nur geringfügig angepasst werden, und die
  Dialektpunkte ergänzt werden.
\end{itemize}



\paragraph{Planung der Applikation}

\begin{itemize}
\itemsep1pt\parskip0pt\parsep0pt
\item
  Grundlegende Idee: Teile die Applikation in zwei Teile, einen mit der
  Karte, einen mit den Alinierungen
\item
  Zur Darstellung der Alinierungen liegt bereits eine
  \href{http://github.com/dighl/prison}{JS-Bibliothek} vor, die es
  ermöglicht, Wörter, die segmentiert vorliegen, koloriert darzustellen.
\item
  Erweiterte Ideen wurden in einem nicht ratsamen
  ``trial-and-error-Vefahren'' entwickelt (planen ist immer besser als
  ausprobieren, aber man plant dann am Ende doch immer viel zu
  wenig\ldots{}).
\end{itemize}


\subsubsection{\texorpdfstring{{Applikation}}{Applikation}}

