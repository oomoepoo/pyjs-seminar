\hyperdef{}{tilda}{}

\subsection{Funktionen im Allgemeinen}

\subsubsection{\texorpdfstring{{Begriffserklärung}}{Begriffserklärung}}

\vspace{0.5cm}\par\noindent\textbf{\href{http://de.wikipedia.org/wiki/Funktion\%20(Mathematik)}{Funktionen\vspace{0.5cm}
in der Mathematik (laut Wikipedia)}}

\begin{quote}
In der Mathematik ist eine Funktion oder Abbildung eine Beziehung
zwischen zwei Mengen, die jedem Element der einen Menge
(Funktionsargument, unabhängige Variable, x-Wert) genau ein Element der
anderen Menge (Funktionswert, abhängige Variable, y-Wert) zuordnet. Das
Konzept der Funktion oder Abbildung nimmt in der modernen Mathematik
eine zentrale Stellung ein; es enthält als Spezialfälle unter anderem
parametrische Kurven, Skalar- und Vektorfelder, Transformationen,
Operationen, Operatoren und vieles mehr.
\end{quote}



\vspace{0.5cm}\par\noindent\textbf{\href{http://de.wikipedia.org/wiki/Funktion\%20(Programmierung)}{Funktionen\vspace{0.5cm}
in der Programmierung (laut Wikipedia)}}

\begin{quote}
Eine Funktion (engl.: function, subroutine) ist in der Informatik die
Bezeichnung eines Programmierkonzeptes, das große Ähnlichkeit zum
Konzept der Prozedur hat. Hauptmerkmal einer Funktion ist es, dass sie
ein Resultat zurückliefert und deshalb im Inneren von Ausdrücken
verwendet werden kann. Durch diese Eigenschaft grenzt sie sich von einer
Prozedur ab, die nach ihrem Aufruf kein Ergebnis/Resultat zurück
liefert. Die genaue Bezeichnung und Details ihrer Ausprägung ist in
verschiedenen Programmiersprachen durchaus unterschiedlich.
\end{quote}



\subsubsection{\texorpdfstring{{Typen von
Funktionen}}{Typen von Funktionen}}

Da entscheidend für Funktionen der Eingabewert und der Rückgabewert
sind, können wir ganz grob die folgenden speziellen Typen von Funktionen
identifizieren:

\begin{itemize}
\itemsep1pt\parskip0pt\parsep0pt
\item
  {Funktionen ohne Rückgabewert (Prozedur)}
\item
  {Funktionen ohne Eingabewert}
\item
  {Funktionen mit einer festen Anzahl von Eingabewerten}
\item
  {Funktionen mit einer beliebigen Anzahl von Eingabewerten}
\end{itemize}
i
\subsection{Funktionen in Python}


\vspace{0.5cm}\par\noindent\textbf{Vorweg}\vspace{0.5cm}

In Python sind Funktionen spezielle Datentypen. Sie unterscheiden sich
von anderen Datentypen wie \textbf{strings} oder \textbf{integers}
dadurch, dass sie zusätzlich aufgerufen werden können (\emph{callable
types}).



\subsubsection{Grundlagen}

\vspace{0.5cm}\par\noindent\textbf{Allgemeine Struktur}\vspace{0.5cm}

\begin{verbatim}
def functionName(
        parameterA,
        ...,
        keywordA='defaultA',
        ...,
        ):
    """
    docString
    """
    functionBody
    return functionValue
\end{verbatim}




\vspace{0.5cm}\par\noindent\textbf{Erstes Beispiel}\vspace{0.5cm}

\begin{verbatim}
def stoneAgeCalculator(intA, intB, calc='+'):
    """
    This is the famous stoneAgeCalculator, a program written by the very first
    men on earth who brought the fire to us and were the first to dance naked
    on the first of May.
    """
    # check for inconsistencies in the input for keyword calc
    if calc not in ['+', '-', '*', '/']:
        raise ValueError('The operation you chose is not defined.')

    # start the calculation, catch errors from input with a simple
    # try-except-structure
    try:
        if calc == '+':
            return intA + intB
        elif calc == '-':
            return intA - intB
        elif calc == '*':
            return intA * intB
        else:
            return intA / intB
    except:
        raise ValueError('No way to operate on your input.')
\end{verbatim}




\vspace{0.5cm}\par\noindent\textbf{Aufrufen und Ausführen}\vspace{0.5cm}

\begin{quote}
Eine Funktion wird aufgerufen, indem der Name der Funktion angegeben
wird, gefolgt von einer Liste aktueller Parameter in Klammern.
\href{http://bibliography.lingpy.org?key=Weigend2008}{(Weigend
2008:144)}
\end{quote}

\begin{verbatim}
>>> a = range(5)
>>> a = int('1')
>>> range(5)
range(0,5)
>>> int('1')
1
>>> float(1)
1.0
>>> print('Hallo Welt!')
Hallo Welt!
>>> list('hallo')
['h', 'a', 'l', 'l', 'o']
\end{verbatim}




\vspace{0.5cm}\par\noindent\textbf{Die Funktionsdokumentation}\vspace{0.5cm}

Für jede Funktion, die in Python definiert wird, steht eine automatische
Dokumentation bereit, welche mit Hilfe der Parameter und der Angaben im
Docstring generiert wird. Die Dokumentation einer Funktion wird mit
Hilfe von \texttt{help(function)} aufgerufen. Mit Hilfe der
Funktionsdokumentation lässt sich leicht ermitteln, welche Eingabewerte
eine Funktion benötigt, wie sie verwendet wird, und welche Werte sie
zurückliefert. Docstrings lassen sich am besten interaktiv einsehen.



\subsubsection{Parameter und Schlüsselwörter}

\vspace{0.5cm}\par\noindent\textbf{Parameter}\vspace{0.5cm}

Mit Hilfe von Funktionen werden Eingabewerte (Parameter) in Ausgabewerte
umgewandelt. Bezüglich der Datentypen von Eingabewerten gibt es in
Python keine Grenzen. Alle Datentypen (also auch Funktionen selbst)
können daher als Eingabewerte für Funktionen verwendet werden.
Gleichzeitig ist es möglich, Funktionen ohne Eingabewerte oder ohne
Ausgabewerte zu definieren. Will man eine beliebige Anzahl an Parametern
als Eingabe einer Funktion verwenden, so erreicht man dies, indem man
der Parameterbezeichnung einen Stern (*) voranstellt. Die Parameter von
Eingabeseite werden innerhalb der Funktion dann automatisch in eine
Liste umgewandelt, auf die mit den normalen Listenoperationen
zugegriffen werden kann.




\vspace{0.5cm}\par\noindent\textbf{Parameter}\vspace{0.5cm}

\begin{verbatim}
>>> def leer():
...     print "leer"
>>> def empty():
...     pass
>>> def gruss1(wort):
...     print wort
>>> def gruss2(wort):
...     return wort
>>> def eins():
...     """Gibt die Zahl 1 aus."""
...     return 1
>>> leer()
leer
>>> empty
... 
>>> empty()
>>> type(empty)

>>> a = eins()
>>> a
1
>>> print bla.func_doc
Gibt die Zahl 1 aus.
>>> def printWords(\*words):
...     for word in words:
...         print(word)
...
>>> printWords('mama','papa','oma',opa')
mama
papa
oma
opa
\end{verbatim}




\vspace{0.5cm}\par\noindent\textbf{Schlüsselwörter}\vspace{0.5cm}

Als Schlüsselwörter bezeichnet man Funktionsparameter, die mit einem
Standardwert belegt sind. Werden diese beim Funktionsaufruf nicht
angegeben, gibt es keine Fehlermeldung, sondern anstelle der fehlenden
Werte wird einfach auf den Standardwert zurückgegriffen. Gleichzeitig
braucht man beim Aufrufen nicht auf die Reihenfolge der Parameter zu
achten, da diese ja durch die Anbindung der Schlüsselwörter vordefiniert
ist. Weist eine Funktion hingegen Schlüsselwörter und Parameter auf, so
müssen die Parameter den Schlüsselwörtern immer vorangehen.




\vspace{0.5cm}\par\noindent\textbf{Schlüsselwörter}\vspace{0.5cm}

\begin{verbatim}
>>> def wind(country, season='summer')
...     """Return the typical wind conditions for a given country."""
...     if country == "Germany" and season == 'summer':
...         print("There's strong wind in",country)
...     else:
...         print("This part hasn't been programmed yet.")
...
>>> wind('Germany'):
There's strong wind in Germany.
>>> wind('Germany,season='winter')
This part hasn't been programmed yet.
\end{verbatim}



\subsubsection{\texorpdfstring{{Spezialfälle}}{Spezialfälle}}

\vspace{0.5cm}\par\noindent\textbf{Prozeduren}\vspace{0.5cm}

\begin{quote}
Prozeduren sind Funktionen, die den Wert \textbf{None} zurückgeben.
Fehlt in einer Funktionsdefinition die \textbf{return}-Anweisung, wird
automatisch der Wert \textbf{None} zurückgegeben.
\href{http://bibliography.lingpy.org?key=Weigend2008}{(Weigend 2008:
155)}
\end{quote}

\begin{verbatim}
>>> def quak(frosch=''):
...     print("quak")
...
>>> quak()
quak
>>> a = quak()
quak
>>> type(a)
\end{verbatim}



\vspace{0.5cm}\par\noindent\textbf{Rekursive Funktionen}\vspace{0.5cm}

\begin{quote}
Rekursive Funktionen sind solche, die sich selbst aufrufen.
\href{http://bibliography.lingpy.org?key=Weigend2008}{(Weigend 2008:
151)}
\end{quote}

\begin{verbatim}
>>> def factorial(number):
...     """
...     Aus: http://www.dreamincode.net/code/snippet2800.htm
...     """
...     if number == 0:
...         return 1
...     else:
...         value = factorial(n-1)
...         result = n * value
...         return result
...
>>> factorial(4)
24
\end{verbatim}



\vspace{0.5cm}\par\noindent\textbf{Globale und lokale Variablen}\vspace{0.5cm}

Es muss bei Funktionen zwischen lokalen und globalen Variablen
unterschieden werden. Lokale Variablen haben nur innerhalb einer
Funktion ihren Gültigkeitsbereich. Globale Variablen hingegen gelten
auch außerhalb der Funktion. Will man mit Hilfe einer Funktion eine
Variable, die global, also außerhalb der Funktion deklariert wurde,
verändern, so muss man ihr das \emph{Keyword} \textbf{global}
voranstellen. Dies sollte man jedoch nach Möglichkeit vermeiden, da dies
schnell zu Fehlern im Programmablauf führen kann. Lokale und globale
Variablen sollten nach Möglichkeit getrennt werden.



\vspace{0.5cm}\par\noindent\textbf{Globale und lokale Variablen}\vspace{0.5cm}

\begin{verbatim}
>>> s = 'globaler string'
>>> def f1():
...     print(s)
...
>>> def f2():
...     s = 'lokaler string'
...     print(s)
...
>>> f1()
globaler string
>>> f2()
lokaler string
>>> def f3():
...     global s
...     s = 'lokaler string'
...     print(s)
>>> s = 'globaler string'
>>> f3()
lokaler string
>>> print(s)
lokaler string
\end{verbatim}

\subsection{Funktionen in JavaScript}

\subsubsection{Grundlagen}

\vspace{0.5cm}\par\noindent\textbf{Allgemeines vorweg}\vspace{0.5cm}

Wie in Python, so sind auch in JavaScript Funktionen spezielle
Datentypen.

\vspace{0.5cm}\par\noindent\textbf{Allgemeine Struktur}\vspace{0.5cm}

\begin{verbatim}
function functionName (
  parameterA,
  parameterB,
  parameterC,
  ...
  ) {

  functionBody;

  return functionValue;
}
\end{verbatim}

\vspace{0.5cm}\par\noindent\textbf{Erstes Beispiel: Der Steinzeittaschenrechner}\vspace{0.5cm}

\begin{verbatim}
function stoneAgeCalculator(intA, intB, calc) {
  /* This is a very famous StoneAge Calculator. */

  /* check for active values for type */
  if (typeof calc == 'undefined') {
    calc = "+";
  }
  /* check for correct values for calc */
  else if(["+", "-", "*", "/"].indexOf(calc) == -1) {
    return false;
  }

  /* return the stuff */
  if (calc == '+') {return intA + intB;}
  else if (calc == '-') {return intA - intB;}
  else if (calc == '*') {return intA * intB;}
  else if (calc == '/') {return intA / intB;}
  return false;
}
\end{verbatim}




\vspace{0.5cm}\par\noindent\textbf{Erstes Beispiel: Der Steinzeittaschenrechner}\vspace{0.5cm}



\subsubsection{\texorpdfstring{{Parameter und
Schlüsselwörter}}{Parameter und Schlüsselwörter}}

\vspace{0.5cm}\par\noindent\textbf{Parameter und Schlüsselwörter}\vspace{0.5cm}

Die Handhabung von Parametern und Schlüsselwörtern ist nicht so leicht
in JavaScript wie in Python. Schlüsselwörter gibt eigentlich gar nicht.
Das heißt leider auch, dass man keine Standardwerte für bestimmte
Parameter annehmen kann, was das Programmieren ein wenig umständlich
macht. Um ähnliche Funktionalitäten wie in Python zu erlangen, behilft
man sich meist mit Ersatzkonstruktionen.




\vspace{0.5cm}\par\noindent\textbf{Parameter und Schlüsselwörter}\vspace{0.5cm}

\begin{verbatim}
function wind(country, season) {
  /* Return the typical wind conditions for a country */

  /* carry out type check of season, to set the basic value */
  if (typeof season == "undefined") {
    season = "summer";
  }

  if (country == "Germany" and season == "summer") {
    return "There's strong wind in "+country+".";
  }
  else {
    return "This part hasn't been programmed yet.";
  }
}
\end{verbatim}




\vspace{0.5cm}\par\noindent\textbf{Parameter und Schlüsselwörter}\vspace{0.5cm}



\subsubsection{\texorpdfstring{{Spezialfälle}}{Spezialfälle}}

\vspace{0.5cm}\par\noindent\textbf{Prozeduren und Rekursive Funktionen}\vspace{0.5cm}

Prozeduren und rekursive Funktionen können in JavaScript genauso
verwendet werden wie in Python. Tendentiell werden sehr viel mehr
Prozeduren in JavaScript verwendet, da das Auslösen von Aktionen mit
Hilfe der HTML-Buttons und der ``onclick''-Anweisung am leichtesten
geregelt werden kann.




\vspace{0.5cm}\par\noindent\textbf{Globale und lokale Variablen}\vspace{0.5cm}

JavaScript unterscheidet nicht direkt zwischen globalen und lokalen
Variablen: Alles, was auf einer jeweils höheren Ebene definiert wurde,
kann auch auf einer niedrigeren Ebene verwendet werden und umgekehrt.
Das Schlüsselwort \textbf{var} kann jedoch verwendet werden, um zu
gewährleisten, dass Variablen nur innerhalb ihrer jeweiligen Ebene
Geltung haben. Sicherheitshalber sollte man daher vor jede
Variablendeklaration das Schlüsselwort \textbf{var} setzen.




\vspace{0.5cm}\par\noindent\textbf{Globale und lokale Variablen}\vspace{0.5cm}

\begin{verbatim}
> s = 'globaler string'
> function f1(){console.log(s)}
> function f2(){s = 'lokaler string'; console.log(s);}
> f1()
'globaler string'
> f2()
'lokaler string'
> function f3() { var s; s = 'lokaler string'; console.log(s);}
> s = 'globaler string';
> f3()
lokaler string
> s 
globaler string
\end{verbatim}

